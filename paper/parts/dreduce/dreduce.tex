\clearpage
\section{D-Reducibility}
\label{sec:dreduce}

We have seen the reducibility of configurations on $R_4$ and $R_5$. Naturally, we might try to investigate the reducibility of configurations on $R_6$ and beyond. However, as we have seen in the increase in complexity for the 1-reducibility on $R_5$, this will likely be a very tough problem. The difficulty of this problem lies in the fact the we try to prove the reducibility of many configurations on $R_n$ \textit{at once}. However, it is much easier to analyze individual configurations. This why special forms of reducibility were created to analyse such individual configurations. Among which D-reducibility, which is inspired by $k$-reducibility.

\subsection{Definition with the Birkhoff Diamond}

We will be working with an example, the \textit{Birkhoff Diamond}  ($\text{Bir}\Diamond$), which is a configuration on $R_6$ with 4 vertices in the core. See Figure \ref{fig:diamond}.

\begin{figure}[!ht]
    \centering
    \begin{tikzpicture}[scale=1.5, mid arrow/.style={
        postaction={ decorate, decoration={ markings, mark=at position 0.6 with { \arrow[black]{>>} } } } }]
        \node[circle, fill, scale=0.015cm, opacity=0.2] (l1) at (-2, 0) { };
        \node[circle, fill, scale=0.015cm, opacity=0.2] (l2) at (-1, 1) { };
        \node[circle, fill, scale=0.015cm] (l3) at (-1, 0) {};
        \node[circle, fill, scale=0.015cm, opacity=0.2] (l4) at (-1, -1) {};

        \node[circle, fill, scale=0.015cm, opacity=0.2] (r1) at (2, 0) {};
        \node[circle, fill, scale=0.015cm, opacity=0.2] (r2) at (1, 1) {};
        \node[circle, fill, scale=0.015cm] (r3) at (1, 0) {};
        \node[circle, fill, scale=0.015cm, opacity=0.2] (r4) at (1, -1) {};

        \node[circle, fill, scale=0.015cm] (c1) at (0, 0.5) {};
        \node[circle, fill, scale=0.015cm] (c2) at (0, -0.5) {};
        \node (core) at (-0.65, 0.45) { $\core$ };
        \node (ring) at (-1.7, 0.7) { $R_6$ };

        \draw[opacity=0.2] (l1) -- (l2) -- (r2) -- (r1) -- (r4) -- (l4);
        \draw [mid arrow, opacity=0.3] (l1) -- (l4);
        \draw[opacity=0.2] (l1) -- (l3);
        \draw[opacity=0.2] (l2) -- (l3) -- (l4);
        \draw[opacity=0.2] (l2) -- (c1);
        \draw (c1) -- (l3) -- (c2);
        \draw[opacity=0.2] (c2) -- (l4);
        \draw (c1) -- (c2);
        \draw[opacity=0.2] (r2) -- (c1);
        \draw (c1) -- (r3) -- (c2);
        \draw[opacity=0.2] (c2) -- (r4);
        \draw[opacity=0.2] (r2) -- (r3) -- (r4);
        \draw[opacity=0.2] (r1) -- (r3);
    \end{tikzpicture}
    \caption{The Birkhoff Diamond $\confg = \bir$ with the core highlighted. }
    \label{fig:diamond}
\end{figure}

An individual configuration on a ring is fully defined by its core $\core$ and the amount of edges each vertex of $\core$ has with the ring $R$. The Birkhoff Diamond is uniquely determined by the four vertices in the middle (its core) and the requirement that each vertex of $\core$ has a total of 5 edges in $\confg$. For vertices in the core, the distinction between various heads of a vertex indicates this degree in $\confg$. This is a convention set by Heesch \cite{heesch} in 1969. Four such vertex-heads are given below.

\begin{equation*}
    \begin{matrix}
    \bullet &\quad \longrightarrow \quad &\deg(v)=5 \\
    \cdot &\quad \longrightarrow \quad & \deg(v)=6 \\
    \circ &\quad \longrightarrow \quad & \deg(v)=7 \\
    \square &\quad \longrightarrow \quad  & \deg(v)=8 \\
    \end{matrix}
\end{equation*}

We will show that $\bir$ is 0-reducible by a careful investigation of all the ring colorings of $\bir$ and $R_6$. Since we know exactly what the graph is of $\bir$, we can write down all its valid ring colorings. We have provided all the ring colorings of $R_6$ and $\Phi(\bir)$ in Figure \ref{table:colsring6}.

\needspace{4cm}
\begin{figure}[!ht]
    \centering
    \begin{tabular}{ cccc }
        $\Phi(6) $ & \\
        \hline
        ababab & abacbd & abcadc &  \cellcolor{g0} abcdab \\
        \cellcolor{g0} ababac &  \cellcolor{g0} abacdb &  \cellcolor{g0} abcbab & abcdac \\
        \cellcolor{g0} ababcb &  \cellcolor{g0} abacdc &  \cellcolor{g0} abcbac & abcdad \\
        ababcd & abcabc &  \cellcolor{g0} abcbad & abcdbc \\
        abacab & abcabd & abcbcb & abcdbd \\
        \cellcolor{g0} abacac &  \cellcolor{g0} abcacb &  \cellcolor{g0} abcbcd &  \cellcolor{g0} abcdcb \\
        abacad &  \cellcolor{g0} abcacd & abcbdb &  \cellcolor{g0} abcdcd \\
        \cellcolor{g0} abacbc &  \cellcolor{g0} abcadb & abcbdc \\
        \hline
        31 & \\
    \end{tabular}
    \caption{The set $\Phi(6)$ and the valid colorings $\Phi(\bir)$ highlighted. }
    \label{table:colsring6}
\end{figure}

To show that $\bir$ is 0-reducible, we must show that all those invalid ring colorings can be fixed using Kempe-chains to become valid. If we let $M+R_6$ be arbitrary, then we can expect any ring coloring of $R_6$. Let us consider the coloring $ababab$ for example. Suppose that $\chain{v_4}{v_6}{bd}$. This implies the following colorings.

\begin{equation}
    \begin{aligned}
    \scheme{a,b,a,b,a,b}{46d} &\compat ababcb\\
    \scheme{a,b,a,b,a,b}{46d-} &\compat ababad = ababac.
    \end{aligned}
\end{equation}

Because the invalid coloring $ababab$ can be reconfigured into a valid coloring with only one Kempe-chain flip, we say that the coloring $ababab$ \textit{implies} the set of colorings

\begin{equation}
    ababab \compat \{ ababcb, ababac \}.
\end{equation}

\begin{definition}
    For a coloring $x$, we have $x \compat \II$ if for every scheme $x^\star$ of $x$, we have $x^\star \compat y \in \II$.
\end{definition}

\begin{definition}
    For two set of colorings $\I$ and $\II$, we $\I \compat \II$ if for every $x \in \I$, we have $x \compat \II$. 
\end{definition}

The set of all colorings of $R_6$ that require one Kempe-chain flip to be reconfigured in the same way as $ababab$ is called the 1-implying set of $\bir$.

\begin{equation}
    \Phi_1(\bir) = \;\; \Phi_0(\bir)\; \cup \; \{ ababab, \;ababcd, \;abacab, \;abcbcb, \;abcdad \}.
\end{equation}

This set is the largest that satisfies $\Phi_1(\bir) \compat \Phi_0(\bir) = \Phi(\bir)$.  We may repeat what we did for $\Phi_1(\bir)$ to obtain sets of colorings that require 2, 3 and more Kempe-chain flips to become a valid coloring. This higher-order implication between sets of colorings is called \textit{n-implication}.

\begin{equation}
    \Phi_5(\bir) \compat \Phi_4(\bir) \compat \Phi_3(\bir) \compat \ldots \compat \Phi(\bir).
\end{equation}

\begin{definition}
    For two sets of colorings $\I$ and $\II$, we have $\I \ncompat{n} \II$ if there exist sets $B_i$ for $0 < i < n$ such that
    \begin{equation}
        I \compat B_{n-1}, \quad B_i \compat B_{i-1} \quad\text{and}\quad B_1 \compat \II.
    \end{equation}
\end{definition}

\begin{definition}
    The $n$-implying set $\Phi_n(\confg)$ of a configuration $\confg$ is the largest set of ring colorings such that $\Phi_n(C) \ncompat{n} \Phi_0(\confg) = \Phi(\confg)$. 
\end{definition}

The set $\Phi_5(\bir)$ satisfies $\Phi_5(\bir) \ncompat{5} \Phi_0(\bir)$. Let us find all the $n$-implying sets of $\bir$. We say that colorings in these sets are \textit{fixable}, because they can always be reconfigured to a valid coloring with Kempe-chains. For D-reducibility, we want all ring colorings to be fixable. All $n$-implying sets together must cover all ring colorings. This is precisely the case with the Birkhoff Diamond.

\needspace{2cm}
\begin{figure}[!ht]
    \centering
    \begin{tabular}{ ccccccc }
        $\Phi_0(\bir) $ & & $\Phi_1$ & $\Phi_2$ & $\Phi_3$ & $\Phi_4$ & $\Phi_5$ \\
        \hline
        ababac & abcadb & ababab & abacad & abacbd & abcabd & abcabc \\
        ababcb & abcbab & ababcd & abcbdb & abcbdc & abcadc & \\
        abacac & abcbac & abacab &        & abcdac & abcdbc & \\
        abacbc & abcbad & abcbcb &        & abcdbd &        & \\
        abacdb & abcbcd & abcdad &        &        &        & \\
        abacdc & abcdab \\
        abcacb & abcdcb \\
        abcacd & abcdcd \\
        \hline
        16 & & 5 & 2 & 4 & 3 & 1 \\
    \end{tabular}
    \caption{ All $n$-implying sets of $\bir$. The largest covers all 31 ring colorings of $R_6$. }
    \label{table:diamondmap}
\end{figure}

\begin{definition}
    The \emph{max-implying} set $\overline{\Phi}(\confg)$ of a configuration $\confg$ is the largest $n$-implying set  $\Phi_n(\confg)$.
\end{definition}

\begin{definition}
    A coloring $x \in \maxi{\confg}$ is called a \emph{fixable} ring coloring of $\confg$.
\end{definition}

\begin{definition}
    A configuration $\confg$ on $R_n$ is D-reducible if $\overline{\Phi}(\confg) = \Phi(n)$.
\end{definition}

\begin{figure}[!ht]
    \centering
    \begin{tikzpicture}[scale=1.0]
        \draw (0, 0) ellipse (3cm and 1.8cm);
        \draw (0, -0.3) ellipse (2cm and 1.2cm);
        \draw[fill opacity=0.4, pattern=north east lines] (0, -0.6) ellipse (1.2cm and 0.6cm);

        \node at (0.0, -0.6) { $\Phi_0(C)$ };
        \node at (0.0, 0.35) { $\Phi_1(C)$ };
        \node at (0, 1.35) { $\Phi(n) = \overline{\Phi}(C)$ };
    \end{tikzpicture}

    \caption{Every ring coloring is fixable for a D-reducible configuration.}
\end{figure}
