\subsection{0-reducibility of the ring $R_4$}

We will show that all configurations on the ring $R_4$ are reducible without the need for a reducer. This proof is similar to what we did in the five color theorem.

\begin{theorem}
    The ring $R_4$ is 0-reducible.
\end{theorem}
\begin{proof}

Let the planar graph $M+\confg$ with configuration $\confg = \core + R_4$ be arbitrary. Since we set $k=0$, we will try reducing to $M+R_4$ and $\confg$. For convenience, let the valid ring colorings of these two graphs be represented by the sets $\I$ and $\II$. To gain an insight into which colorings we can expect in these sets, we sketched the situation in Figure \ref{fig:ring4tut}.

\begin{equation}
    \I = \Phi(M+R_4) \quad \text{and} \quad \II = \Phi(\confg).
\end{equation}

\begin{figure}[!ht]
    \centering
    \begin{tikzpicture}[mid arrow/.style={
        postaction={ decorate, decoration={ markings, mark=at position 0.6 with { \arrow[black]{>>} } } } }]
        \draw[fill=white] (-0.5, 0) ellipse (2cm and 1.5cm);
        \node (m) at (-1.7, 0) {$M$};
        \node at (-0.75, 0.75) {$R_4$};

        \node[circle, fill, scale=0.015cm] (l1) at (0, 0.9) { };
        \node[circle, fill, scale=0.015cm] (l2) at (0.9, 0) { };
        \node[circle, fill, scale=0.015cm] (l3) at (0, -0.9) {};
        \node[circle, fill, scale=0.015cm] (l4) at (-0.9, 0) {};

        \draw[mid arrow] (l1) -- (l2);
        \draw (l2) -- (l3) -- (l4) -- (l1);
    \end{tikzpicture}
    \begin{tikzpicture}[mid arrow/.style={
        postaction={ decorate, decoration={ markings, mark=at position 0.6 with { \arrow[black]{>>} } } } }]
        \draw[opacity=0] (-0.5, 0) ellipse (2cm and 1.5cm);
        \node at (-0.75, 0.75) {$R_4$};
        \node[inner sep=1mm] (c) at (0, 0) {$\core$};

        \node[circle, fill, scale=0.015cm] (l1) at (0, 0.9) { };
        \node[circle, fill, scale=0.015cm] (l2) at (0.9, 0) { };
        \node[circle, fill, scale=0.015cm] (l3) at (0, -0.9) {};
        \node[circle, fill, scale=0.015cm] (l4) at (-0.9, 0) {};

        \draw[mid arrow] (l1) -- (l2);
        \draw (l2) -- (l3) -- (l4) -- (l1);
        \draw[opacity=0.2] (l1) -- (c);
        \draw[opacity=0.2] (l2) -- (c); 
        \draw[opacity=0.2] (l3) -- (c);
        \draw[opacity=0.2] (l4) -- (c);
    \end{tikzpicture}
    \caption{The reductions $M+R_4$ and $\core+R_4$ ($\confg$). What can we say about their possible ring colorings $\I$ and $\II$?}
    \label{fig:ring4tut}
\end{figure}

You might have noticed that both reductions contain the plain ring $R_4$ with no other vertices on one side. Therefore, as we have shown in Theorem \ref{thm:ringsarered} about plain ring reducibility, we may contract any two non-neighboring vertices to obtain further reductions. We wont sketch these contracted graphs here, but in the context of ring colorings in $\I$ and $\II$, we will obtain colorings of $R_4$ with two contracted vertices colored the same. Because there are two ways to contract vertices on $R_4$ (the two diagonals), we are guaranteed of the following two colorings in $\I$ and $\II$.

\begin{equation}
    \left\{\begin{matrix}
        abab \;\;\text{or}\;\; abac, \\
        baba \;\;\text{or}\;\; baca
    \end{matrix}\right\} \subset I,II.
\end{equation}

Let us evaluate every pair of possibilities to see if there is a common coloring. First note that $abab=baba$ by definition of equality between ring colorings (Definition \ref{def:coleq}). Then, there remain 3 possible sets for $\I$ and $\II$.

\needspace{2cm}
\begin{equation}
    \circled{1} = \{ abab \}, \quad \circled{2} = \left\{ \begin{matrix}abab \\ baca\end{matrix} \right\}, \quad \circled{3} = \left\{ \begin{matrix}abac \\ baca \end{matrix} \right\}.
\end{equation}

For subsets of $\I$ and $\II$, $\circled{1}$ and $\circled{2}$ implies that $abab$ is the common coloring. $\circled{2}$ and $\circled{3}$ implies that $baca$ is the common coloring. To handle the third case, let $\{ abab \} \subset \I$ and $\{ abac, baca \} \subset \II$. Similar to what we did with the five color theorem, suppose that in the coloring $\I(abab)$ we have $\chain{v_1}{v_3}{ad}$. In both cases we obtain

\begin{equation}
    \begin{aligned}
        \I(abab) &= \scheme{a,b,a,b}{13d} \compat \II(abac) \\
        \I(abab) &= \scheme{a,b,a,b}{13d-} \compat \I(abcb) = \II(baca).
    \end{aligned}
\end{equation}

In any case, we obtain a common ring coloring between $\I$ and $\II$. Therefore the ring $R_4$ is 0-reducible.
\end{proof}