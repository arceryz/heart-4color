\subsection{Reducibility of configurations on $R_4$}

Using just the power of schemes and Kempe-chains, we can already tackle the reducibility of any configuration on $R_4$. This proof serves as an excellent introduction into applying schemes and Kempe-chains, just like in the five color theorem.

\begin{theorem}
    Every configuration $\confg$ on $R_4$ is reducible in every planar graph.
\end{theorem}
\begin{proof}

Let the planar graph $M+\confg$ and configuration $\confg = \core + R_4$ be arbitrary. Let us denote the sets of ring colorings of the reductions $M+R$ and $\confg$ by $\I$ and $\II$ respectively. We will refer to a coloring in one of these sets by prepending the set name, i.e $\I(abab)$ means coloring $abab$ in set $\I$. The situation is sketched in Figure \ref{fig:ring4tut}. 

\begin{equation}
    \I = \Phi(M+R_4) \quad \text{and} \quad \II = \Phi(\confg).
\end{equation}

\begin{figure}[!ht]
    \centering
    \begin{tikzpicture}[mid arrow/.style={
        postaction={ decorate, decoration={ markings, mark=at position 0.6 with { \arrow[black]{>>} } } } }]
        \draw[fill=white] (-0.5, 0) ellipse (2cm and 1.5cm);
        \node (m) at (-1.7, 0) {$M$};
        \node at (-0.75, 0.75) {$R_4$};

        \node[circle, fill, scale=0.015cm] (l1) at (0, 0.9) { };
        \node[circle, fill, scale=0.015cm] (l2) at (0.9, 0) { };
        \node[circle, fill, scale=0.015cm] (l3) at (0, -0.9) {};
        \node[circle, fill, scale=0.015cm] (l4) at (-0.9, 0) {};

        \draw[mid arrow] (l1) -- (l2);
        \draw (l2) -- (l3) -- (l4) -- (l1);
    \end{tikzpicture}
    \begin{tikzpicture}[mid arrow/.style={
        postaction={ decorate, decoration={ markings, mark=at position 0.6 with { \arrow[black]{>>} } } } }]
        \draw[opacity=0] (-0.5, 0) ellipse (2cm and 1.5cm);
        \node at (-0.75, 0.75) {$R_4$};
        \node[inner sep=1mm] (c) at (0, 0) {$\core$};

        \node[circle, fill, scale=0.015cm] (l1) at (0, 0.9) { };
        \node[circle, fill, scale=0.015cm] (l2) at (0.9, 0) { };
        \node[circle, fill, scale=0.015cm] (l3) at (0, -0.9) {};
        \node[circle, fill, scale=0.015cm] (l4) at (-0.9, 0) {};

        \draw[mid arrow] (l1) -- (l2);
        \draw (l2) -- (l3) -- (l4) -- (l1);
        \draw[opacity=0.2] (l1) -- (c);
        \draw[opacity=0.2] (l2) -- (c); 
        \draw[opacity=0.2] (l3) -- (c);
        \draw[opacity=0.2] (l4) -- (c);
    \end{tikzpicture}
    \caption{The reductions $M+R_4$ and $\core+R_4$ ($\confg$).}
    \label{fig:ring4tut}
\end{figure}

Both reductions contain the ring $R_4$. Therefore, by Theorem \ref{thm:ringsarered}, we may further reduce these graphs by contracting any two opposing vertices of $R_4$. Because there are only 2 ways to contract non-neighboring vertices of $R_4$, we will obtain two guaranteed colorings in $\I$ and $\II$. In each of them, the contracted vertices are colored the same.

\begin{equation}
    \left\{\begin{matrix}
        abab \;\;\text{or}\;\; abac, \\
        baba \;\;\text{or}\;\; baca
    \end{matrix}\right\} \subset I,II.
\end{equation}

First note that $abab = baba$ are the same coloring. The possibilities result in a total of 4 different sets of guaranteed colorings for $\I$ and $\II$.

\needspace{2cm}
\begin{equation}
    \circled{1} = \{ abab \}, \quad \circled{2} = \left\{ \begin{matrix}abab \\ baca\end{matrix} \right\}, \quad \circled{3} = \left\{ \begin{matrix}abac \\ baba \end{matrix} \right\}, \quad \circled{4} = \left\{ \begin{matrix}abac \\ baca \end{matrix} \right\}.
\end{equation}

All pairs of sets except $\circled{1}$ and $\circled{4}$ already have a common coloring. Therefore, let $\{ abab \} \subset \I$ and $\{ abac, baca \} \subset \II$. Now, we make a case distinction whether the chain $\chain{v_1}{v_3}{ad}$ exists in $\I(abab)$ or not.

\begin{equation}
    \begin{aligned}
        \I(abab) &= \scheme{a,b,a,b}{13d} \compat \II(abac) \\
        \I(abab) &= \scheme{a,b,a,b}{13d-} \compat \I(abdb) = \II(baca).
    \end{aligned}
\end{equation}

In any case, we obtain a common ring coloring between $\I$ and $\II$. Therefore the ring $R_4$ is 0-reducible.
\end{proof}