\section*{Summary}
\label{sec:summary}

The four color theorem states that every map can be colored with four colors in such a way, that two neighboring regions receive different colors. Such a coloring is desired for a world map, because it becomes easy to tell two neighboring regions apart. It has been observed by many map makers that four colors suffice. The problem was first formulated by Francis Guthrie in 1852 while coloring the map of England. He brought the problem to his brother Frederick Guthrie, who in turn brought the problem to his mathematics lecturer Augustus De Morgan. 

At the heart of the famous proof of the four color theorem by Appel and Haken in 1976 \cite{appel} is shown that any map contains an arrangment of regions that can be removed and recolored later. These are called \textit{reducible configurations}. The four and five color theorem both used the theory of reducible configurations. There are three forms of reducibility of a configuration used in these proofs.

\begin{enumerate}
\item \textit{k-reducibility} shows that configurations with a boundary ring on less than 6 regions can be reduced under certain conditions. This is done by proving that the interior and exterior can be independently colored such that the colors on the bordering ring are the same. This justifies removing the configuration from the map and adding it back after coloring the smaller map.

\item \textit{D-reducibility} builds upon $k$-reducibility for configurations on rings of 6 regions and more. This technique excels at proving reducibility for individual configurations. A configuration is D-reducible if every ring coloring can be fixed to become a valid ring coloring of the configuration. \textit{Kempe-chains} are used to reconfigure invalid ring colorings to valid ones. The Birkhoff Diamond is the most famous example for D-reducibility.

\item \textit{C-reducibility} improves upon D-reducibility in case there are unfixable ring colorings. It avoids these unfixable colorings by replacing the configuration with a smaller map called a \textit{reducer}. The ring colorings of the reducer put constraints on the ring colorings of the configuration. These contraints are selected such that the constrained colorings are all fixable. This avoids the unfixable colorings, making the configuration reducible.
\end{enumerate}

The Bernhart Diamond is C-reducible with a valid reducer. However, two of the constrained colorings from the reducer are unfixable. This turns out to be a flaw in either D-reducibility, or the implementation of D-reducibility by John. P. Steinberger \cite{johnp}. These two colorings depend on another problem coloring called a \textit{symmetry fault}. A symmetry fault is a coloring that is unfixable, but whose symmetry \textit{is} fixable. Therefore, both symmetries must be fixable. We have not uncovered the cause of these faults, but Bernhart \cite{bernhart} proved that in case of the Bernhart Diamond, they are not a problem.

The first proof of the four color theorem had a set of 1478 configurations to check. An improvement was made by Neil Robertson et al. \cite{thomas} in 1996 who reduced the set to only 633 configurations that are either C or D-reducible. Lastly, an improvement was made by John P. Steinberger in 2009 who used only D-reducibility at the cost of having to check 2822 configurations. It seems that using more advanced forms of reducibility reduces the amount of configurations on ring 6 and above. This hints that there might be an all-encompassing form of reducibility that uses the least number of configurations, which would be \textit{The Heart of the Four Color Theorem}.