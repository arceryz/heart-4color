\section{D-Reducibility}
\label{sec:dreduce}

We have seen that the ring $R_4$ is 0-reducible and the that the ring $R_5$ is 1-reducible. This essentially provides us with an two infinite classes of reducible configurations. However, it is not yet guaranteed that every planar graph contains such a $k$-reducible configuration on $R_4$ or $R_5$. Therefore, we must still look for more reducible configurations.

Naturally, we might try to examine the $k$-reducibility of $R_6$ and beyond. However, as we have seen in the increase in complexity for the 1-reducibility of $R_5$, this is a very tough problem. There are much more cases to consider for the ring $R_6$ than for $R_4$ and $R_5$. However, the difficulty of $k$-reducibility lies in the fact the we try to prove the reducibility of \textit{all} configurations $\confg$ on $R_n$ at once. Individual configurations are much easier to examine. 

Let us introduce the idea of D-reducibility, by working with an example. The very first, smallest and most famous configuration used in the proof of the four color theorem is the \textit{Birkhoff Diamond}  ($\text{Bir}\Diamond$). It is a configuration on $R_6$ with 4 vertices in the core.

\begin{figure}[!ht]
    \centering
    \begin{tikzpicture}[scale=1.5, mid arrow/.style={
        postaction={ decorate, decoration={ markings, mark=at position 0.6 with { \arrow[black]{>>} } } } }]
        \node[circle, fill, scale=0.015cm, opacity=0.2] (l1) at (-2, 0) { };
        \node[circle, fill, scale=0.015cm, opacity=0.2] (l2) at (-1, 1) { };
        \node[circle, fill, scale=0.015cm] (l3) at (-1, 0) {};
        \node[circle, fill, scale=0.015cm, opacity=0.2] (l4) at (-1, -1) {};

        \node[circle, fill, scale=0.015cm, opacity=0.2] (r1) at (2, 0) {};
        \node[circle, fill, scale=0.015cm, opacity=0.2] (r2) at (1, 1) {};
        \node[circle, fill, scale=0.015cm] (r3) at (1, 0) {};
        \node[circle, fill, scale=0.015cm, opacity=0.2] (r4) at (1, -1) {};

        \node[circle, fill, scale=0.015cm] (c1) at (0, 0.5) {};
        \node[circle, fill, scale=0.015cm] (c2) at (0, -0.5) {};

        \draw[opacity=0.2] (l1) -- (l2) -- (r2) -- (r1) -- (r4) -- (l4);
        \draw [mid arrow, opacity=0.3] (l1) -- (l4);
        \draw[opacity=0.2] (l1) -- (l3);
        \draw[opacity=0.2] (l2) -- (l3) -- (l4);
        \draw[opacity=0.2] (l2) -- (c1);
        \draw (c1) -- (l3) -- (c2);
        \draw[opacity=0.2] (c2) -- (l4);
        \draw (c1) -- (c2);
        \draw[opacity=0.2] (r2) -- (c1);
        \draw (c1) -- (r3) -- (c2);
        \draw[opacity=0.2] (c2) -- (r4);
        \draw[opacity=0.2] (r2) -- (r3) -- (r4);
        \draw[opacity=0.2] (r1) -- (r3);
    \end{tikzpicture}
    \caption{The Birkhoff Diamond $\confg = \bir$ with the core highlighted. }
    \label{fig:diamond}
\end{figure}

We highlighted the core $\core$ of the Birkhoff Diamond here, to explain the format of the \textit{unavoidable set of reducible configurations} found in the original proofs. Every (triangular) configuration with a ring is uniquely determined by its core $\core$ and the amount of edges that a vertex of the core $\core$ has in $\confg$.

For example, the Birkhoff Diamond is uniquely determined by the four vertices in the middle and the requirement that $\deg_\confg(v) = 5$ for each vertex in the core. To save space when storing configuations on paper or digitally, only this information of the core is actually needed.

We will show that $\bir$ is 0-reducible. Since we know the graph of $\bir$, we can write down all the colorings it can have on the ring. This is the set $\Phi(\bir)$ of 16 colorings. See Figure \ref{table:diamondphi0}.

\needspace{2cm}
\begin{figure}[!ht]
    \centering
    \begin{tabular}{ cccc }
        $\Phi(\bir) $ & \\
        \hline
        ababac & abacdb & abcadb & abcbcd \\
        ababcb & abacdc & abcbab & abcdab \\ 
        abacac & abcacb & abcbac & abcdcb \\
        abacbc & abcacd & abcbad & abcdcd \\
        \hline
        16 & \\
    \end{tabular}
    \caption{The unique ring colorings of $\bir$.}
    \label{table:diamondphi0}
\end{figure}

\needspace{1cm}
Therefore, if we want that $\bir$ is 0-reducible, we must show that any ring coloring of $M+R_6$ falls in the set $\Phi(\bir)$. When we were working with rings, we could only use the set of \textit{guaranteed} colorings of $\confg$. However, with an actual configuration like $\bir$, we know exactly which colorings are present. So we have much more control and precision to prove 0-reducibility.

If we let $M+R_6$ be arbitrary, then we can expect any ring coloring of $R_6$. Thus we must show that every coloring of $\Phi(6)$ can be changed to a coloring of $\Phi(\bir)$. The set $\Phi(6)$ can be seen in Figure \ref{table:colsring6}.

\begin{figure}[!ht]
    \centering
    \begin{tabular}{ cccc }
        $\Phi(6) $ & \\
        \hline
        ababab & abacbd & abcadc &  \cellcolor{g0} abcdab \\
        \cellcolor{g0} ababac &  \cellcolor{g0} abacdb &  \cellcolor{g0} abcbab & abcdac \\
        \cellcolor{g0} ababcb &  \cellcolor{g0} abacdc &  \cellcolor{g0} abcbac & abcdad \\
        ababcd & abcabc &  \cellcolor{g0} abcbad & abcdbc \\
        abacab & abcabd & abcbcb & abcdbd \\
        \cellcolor{g0} abacac &  \cellcolor{g0} abcacb &  \cellcolor{g0} abcbcd &  \cellcolor{g0} abcdcb \\
        abacad &  \cellcolor{g0} abcacd & abcbdb &  \cellcolor{g0} abcdcd \\
        \cellcolor{g0} abacbc &  \cellcolor{g0} abcadb & abcbdc \\
        \hline
        31 & \\
    \end{tabular}
    \caption{All unique ring colorings of $R_6$. The colorings of $\Phi(\bir)$ are highlighted. }
    \label{table:colsring6}
\end{figure}

As you can see, roughly half of the colorings is not directly compatible with $\bir$. Similar to what we did for the 1-reducibility of $R_5$, we will use Kempe-chains to change incompatible colorings to colorings in $\Phi(\bir)$.

Let us consider the coloring $ababab$ for example. Suppose that $\chain{v_4}{v_6}{bd}$. This implies the following colorings.

\begin{equation}
    \begin{aligned}
    \scheme{a,b,a,b,a,b}{46d} &\compat ababcb\\
    \scheme{a,b,a,b,a,b}{46d-} &\compat ababad = ababac.
    \end{aligned}
\end{equation}

Therefore, the coloring $ababab$ can be turned into a compatible coloring with only one chain flip. We say that the coloring $ababab$ \textit{implies} the set of colorings 

\begin{equation}
    ababab \compat \{ ababcb, ababac \}.
\end{equation}

This idea of a coloring implying a set of other colorings lies at the heart of D-reducibility, hence we will define it.

\begin{definition}
    A coloring $x$ implies a set of colorings $\II$ if every scheme $x^\star$ of $x$ implies a coloring $y \in \II$. Write $x \compat \II$.
\end{definition}

\begin{definition}
    A set of colorings $\I$ implies $\II$ if every $x \in \I$ implies $\II$. Write $\I \implies \II$.
\end{definition}

Now, let us find all the colorings of $R_6$ that require one chain flip to become compatible in the same way as $ababab$. This set is called the 1-implying set of $\bir$.

\begin{figure}[!ht]
    \centering
    \begin{tabular}{ cc }
        $\Phi_1(\bir) $ \\
        \hline
        ababab & abcbcb \\
        ababcd & abcdad\\
        abacab \\
        \hline
        5 \\
    \end{tabular}
    \caption{The 1-implying set $\Phi_1(\bir)$.}
    \label{table:diamondphi1}
\end{figure}

This is the largest set that satisfies $\Phi_1(\bir) \compat \Phi(\bir)$. We may repeat what we did for $\Phi_1(\bir)$ to obtain sets of colorings that require 2, 3 and more chain flips to become a coloring in $\Phi(\bir)$.

\begin{equation}
    \Phi_5(\bir) \compat \Phi_4(\bir) \compat \Phi_3(\bir) \compat \ldots \compat \Phi(\bir).
\end{equation}

Let us first define the notion of higher-order implication between sets of colorings, called \textit{n-implication}.

\begin{definition}
    A set of colorings $\I$ $n$-implies a set $\II$ if there exist sets $B_i$ for $0 < i < n$ such that $I \compat B_{n-1}$, $B_i \compat B_{i-1}$ and $B_1 \compat \II$. We write $\I \ncompat{n} \II$.
\end{definition}

Therefore the set $\Phi_5(\bir)$, for example, satisfies $\Phi_5(\bir) \ncompat{5} \Phi_0(\bir)$. This is essentially the definition of the $n$-implying set of $\bir$.

\begin{definition}
    The $n$-implying set $\Phi_n(\confg)$ of a configuration $\confg$ is the largest set of ring colorings such that $\Phi_n(C) \ncompat{n} \Phi_0(\confg) = \Phi(\confg)$. 
\end{definition}

To continue with our example, let us find all the $n$-implying sets of $\bir$. In this case, there are only 6 including $\Phi_0(\bir)$.

\needspace{2cm}
\begin{figure}[!ht]
    \centering
    \begin{tabular}{ ccccccc }
        $\Phi_0(\bir) $ & & $\Phi_1$ & $\Phi_2$ & $\Phi_3$ & $\Phi_4$ & $\Phi_5$ \\
        \hline
        ababac & abcadb & ababab & abacad & abacbd & abcabd & abcabc \\
        ababcb & abcbab & ababcd & abcbdb & abcbdc & abcadc & \\
        abacac & abcbac & abacab &        & abcdac & abcdbc & \\
        abacbc & abcbad & abcbcb &        & abcdbd &        & \\
        abacdb & abcbcd & abcdad &        &        &        & \\
        abacdc & abcdab \\
        abcacb & abcdcb \\
        abcacd & abcdcd \\
        \hline
        16 & & 5 & 2 & 4 & 3 & 1 \\
    \end{tabular}
    \caption{ All $n$-implying sets of $\bir$. Together a total of 31 colorings. }
    \label{table:diamondmap}
\end{figure}

If you count the colorings, you will find that all $n$-implying sets together form 31 colorings. This is exactly the number of ring colorings of $R_6$. Therefore, all colorings of $R_6$ can be made compatible with $\bir$ through chain flips. This is exactly what D-reducibility requires.

\begin{definition}
    The \emph{max-implying} set $\overline{\Phi}(\confg)$ of a configuration $\confg$ is the largest $n$-implying set  $\Phi_n(\confg)$.
\end{definition}

\begin{definition}
    A configuration $\confg$ on $R_n$ is D-reducible if $\overline{\Phi}(\confg) = \Phi(n)$.
\end{definition}

\begin{figure}[!h]
    \centering
    \begin{tikzpicture}[scale=1.0]
        \draw (0, 0) ellipse (3cm and 1.8cm);
        \draw (0, -0.3) ellipse (2cm and 1.2cm);
        \draw[fill opacity=0.4, pattern=north east lines] (0, -0.6) ellipse (1.2cm and 0.6cm);

        \node at (0.0, -0.6) { $\Phi_0(C)$ };
        \node at (0.0, 0.35) { $\Phi_1(C)$ };
        \node at (0, 1.35) { $\Phi(n) = \overline{\Phi}(C)$ };
    \end{tikzpicture}

    \caption{The $n$-implying sets of $\confg$ are increasing in size. If they grow to the set of all ring colorings $\Phi(n)$, the configuration is D-reducible. }
\end{figure}

However, it can occur that $\overline{\Phi}(\confg) \neq \Phi(n)$. This is the case with the ring $R_5$ with a single vertex on the inside, which only supports 3-colorings. To handle (some) of these configurations, a stronger form of D-reducibility was required. This is where we go to C-reducibility.