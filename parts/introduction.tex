\section{Introduction}

The four-color theorem is proven by disproving the existence of minimal counterexamples. Let us begin by defining such a minimal counterexample.
\begin{definition}
A graph $G$ is a minimal counterexample to the four-color theorem if $G$ is not four-colorable but any graph $H$ of lower weight $|V(H)|+|E(H)|$ is.
\end{definition}

All minimal counterexamples we mention will be those of the four-color theorem.
The three theorems to the four-color theorem are then as follows. We leave further definitions to Sections \ref{sec:birkhoff} and \ref{sec:dreduce}.

\begin{theorem}
Minimal counterexamples are Birkhoff graphs.
\end{theorem}

\begin{theorem}
Minimal counterexamples do not contain configurations.
\end{theorem}

\begin{theorem}
Birkhoff graphs contain a configuration.
\end{theorem}

Clearly these three results contradict each other. Minimal counterexamples are proven to have no configurations, but subsequently through Birkhoff graphs, do in fact have configurations. As a consequence, there can not exist any minimal counterexamples and the four-color theorem is true. 

Theorem 1 is proven by hand in Section \ref{sec:birkhoff}. Theorem 2 and 3 require a computer to verify all cases. Birkhoff graphs are also called "internally 6-connected triangulations" in literature, but we will use the former for readability.
